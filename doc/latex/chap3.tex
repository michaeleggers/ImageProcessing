\chapter{Komposition}

Sind die beiden Bilder erst einmal fertig verzerrt worden, so werden
sie nun, wie eingangs beschrieben, durch eine Kreuzblende 
zusammengefügt. Zu beachten ist, dass das Zielbild
in Richtung des Quellbildes gewarpt wurde. Die Bildsequenzen
für die beiden Bilder sehen dementsprechend folgenderma\ss en
aus:

\begin{figure}[htbp]
	\centering


	
	\begin{subfigure}[b]{0.19\textwidth}
		\centering
		\includegraphics[width=\textwidth]{source/src0.jpg} % Replace 'image1' with your image file name
		\caption{}
	\end{subfigure}
		\begin{subfigure}[b]{0.19\textwidth}
		\centering
		\includegraphics[width=\textwidth]{source/src1.jpg} % Replace 'image1' with your image file name
		\caption{}
	\end{subfigure}
		\begin{subfigure}[b]{0.19\textwidth}
		\centering
		\includegraphics[width=\textwidth]{source/src2.jpg} % Replace 'image1' with your image file name
		\caption{}
	\end{subfigure}
		\begin{subfigure}[b]{0.19\textwidth}
		\centering
		\includegraphics[width=\textwidth]{source/src3.jpg} % Replace 'image1' with your image file name
		\caption{}
	\end{subfigure}
		\begin{subfigure}[b]{0.19\textwidth}
		\centering
		\includegraphics[width=\textwidth]{source/src4.jpg} % Replace 'image1' with your image file name
		\caption{}
	\end{subfigure}
		\begin{subfigure}[b]{0.19\textwidth}
		\centering
		\includegraphics[width=\textwidth]{source/src5.jpg} % Replace 'image1' with your image file name
		\caption{}
	\end{subfigure}
			\begin{subfigure}[b]{0.19\textwidth}
		\centering
		\includegraphics[width=\textwidth]{source/src6.jpg} % Replace 'image1' with your image file name
		\caption{}
	\end{subfigure}
	

	
		\caption{Quell- zu Ziel warps}
	\label{fig:sources}
	\end{figure}
	% Repeat the subfigure environment for each image
	% ... (Add more subfigures for the remaining images)
	
	% Example of a second row
\begin{figure}[htbp]
	\centering

	    	
	\begin{subfigure}[b]{0.19\textwidth}
		\centering
		\includegraphics[width=\textwidth]{dst/dest6} % Replace 'image7' with your image file name
		\caption{}
	\end{subfigure}
		\begin{subfigure}[b]{0.19\textwidth}
		\centering
		\includegraphics[width=\textwidth]{dst/dest5} % Replace 'image7' with your image file name
		\caption{}
	\end{subfigure}
		\begin{subfigure}[b]{0.19\textwidth}
		\centering
		\includegraphics[width=\textwidth]{dst/dest4} % Replace 'image7' with your image file name
		\caption{}
	\end{subfigure}
		\begin{subfigure}[b]{0.19\textwidth}
		\centering
		\includegraphics[width=\textwidth]{dst/dest3} % Replace 'image7' with your image file name
		\caption{}
	\end{subfigure}
		\begin{subfigure}[b]{0.19\textwidth}
		\centering
		\includegraphics[width=\textwidth]{dst/dest2} % Replace 'image7' with your image file name
		\caption{}
	\end{subfigure}
		\begin{subfigure}[b]{0.19\textwidth}
		\centering
		\includegraphics[width=\textwidth]{dst/dest1} % Replace 'image7' with your image file name
		\caption{}
	\end{subfigure}
		\begin{subfigure}[b]{0.19\textwidth}
		\centering
		\includegraphics[width=\textwidth]{dst/dest0} % Replace 'image7' with your image file name
		\caption{}
	\end{subfigure}
	% ... (Add subfigures for the second row)
	
		    
	\caption{Ziel- zu Quell warps}
	\label{fig:destinations}
\end{figure}

Die Sequenz in Abbildung \ref{fig:destinations} muss zunächst noch
in ihrer Reihenfolge geändert werden bevor
die Kreuzblende angewendet wird.

\begin{figure}[htbp]
	\centering

		    	
	\begin{subfigure}[b]{0.4\textwidth}
		\centering
		\includegraphics[width=\textwidth]{final/final0} % Replace 'image7' with your image file name
		\caption{}
	\end{subfigure}
	\begin{subfigure}[b]{0.4\textwidth}
		\centering
		\includegraphics[width=\textwidth]{final/final1} % Replace 'image7' with your image file name
		\caption{}
	\end{subfigure}
	\begin{subfigure}[b]{0.4\textwidth}
		\centering
		\includegraphics[width=\textwidth]{final/final2} % Replace 'image7' with your image file name
		\caption{}
	\end{subfigure}
	\begin{subfigure}[b]{0.4\textwidth}
		\centering
		\includegraphics[width=\textwidth]{final/final3} % Replace 'image7' with your image file name
		\caption{}
	\end{subfigure}
	\begin{subfigure}[b]{0.4\textwidth}
		\centering
		\includegraphics[width=\textwidth]{final/final4} % Replace 'image7' with your image file name
		\caption{}
	\end{subfigure}
	\begin{subfigure}[b]{0.4\textwidth}
		\centering
		\includegraphics[width=\textwidth]{final/final5} % Replace 'image7' with your image file name
		\caption{}
	\end{subfigure}
	\begin{subfigure}[b]{0.4\textwidth}
		\centering
		\includegraphics[width=\textwidth]{final/final6} % Replace 'image7' with your image file name
		\caption{}
	\end{subfigure}
	% ... (Add subfigures for the second row)
			    
	\caption{Finale Komposition}
	\label{fig:final}
\end{figure}


