\chapter{Software}

Fuer den Kurs haben wir eine Beispielanwendung, PowerMorph, erstellt.
Fuer das Bauen wird ein CMake file bereitgestellt. Die Software basiert
auf OpenGL 3.0 und ist somit auch auf MacOS lauffaehig (zwar hat Apple
den Support von OpenGL eingestellt, allerdings werden Anwendungen, welche bis
einschl. maximal Version 4.1 von OpenGL verwenden, nach wie vor unterstuetzt).
Ausserdem nutzen wir folgende externe Bibliotheken:

\begin{itemize}
	\item \textbf{SDL2} als Abstraktion zum Betriebssystem fuer Fenster und OpenGL context Erstellung.
	\href{https://github.com/libsdl-org/SDL}{https://github.com/libsdl-org/SDL} 
	\item \textbf{STB image/image write}: Lesen/Schreiben von Bilddateien.
		\href{https://github.com/nothings/stb}{https://github.com/nothings/stb} 
	\item \textbf{GLM}: Mathematik Bibliothek, die gut mit OpenGL zusammenarbeitet. 
			\href{https://github.com/g-truc/glm}{https://github.com/g-truc/glm}
	\item \textbf{Dear ImGUI}: Immediate Mode GUI Bibliothek fuer den Editor.
	\item \textbf{tinyfiledialogs}: Betriebssystemunabhaengige Bibliothek fuer Window-Messages, Oeffnen/Speichern Dialoge.
	\item \textbf{GIF writer by Charlie Tangora}: Speichern der gerenderten Sequenzen als GIF.
				\href{https://github.com/charlietangora/gif-h}{https://github.com/charlietangora/gif-h} 
\end{itemize}

\section{Installation}

Wir nutzen CMake, um Projektdateien
fuer eine IDE bzw. Makefiles zu generieren. 

Nach der Installation ist sollte die Software ohne weiteres Aufrufbar sein. Ein fertiges Binary fuer Windows 64 bit ist auf 
www.nocheinfuegen.de herunterladbar.

\section{Verwenden von PowerMorph}

Die Bedienung ist weitestgehend selbsterklaerend. Im \textbf{Source}-Fenster wird eine Linie fuer ein Feature gezogen. Per
Mausklick wird der Fusspunkt der Linie gesetzt, ein weiterer
Mausklick vervollstaendigt diese. Dabei ist die Richtung der Linie
durch den letzten Klick gegeben. Danach muss
im \textbf{Destination}-Fenster eine weitere Linie erzeugt werden, um das
Paar zu komplettieren. Dieses Vorgehen kann so lange wiederholt
werden, wie gewuenscht. Eine bereits gesetzte Linie kann immer mit
der Tastenkombination CTRL+Z rueckgaengig gemacht werden.
Ist man mit der Definition der Features zufrieden, koennen mit einem
Klick auf den \textbf{MAGIC!}-Button die Morphs von Quell- und Zielbild
generiert und danach ueberblendet werden. Je nach Aufloesung des Bilderpaares
und der Anzahl der gesetzen Linien kann dieser Vorgang ein wenig
dauern. Ist die Berechnung abgeschlossen, so wird ein
neues \textbf{Result}-Fenster geoeffnet. Dort laesst sich das
Resultat begutachten. Die Parameter \textbf{a}, \textbf{b} und
\textbf{p}, um das Gewicht einer Linie zu bestimmen, lassen
sich mit den Slidern im \textbf{Control Panel} festlegen.

\begin{figure}[htb]
	\centering
	\includegraphics[width=1.0\textwidth]{powermorph.png}
	\caption{PowerMorph}
	\label{fig:powermorph}
\end{figure}

\section{Ausblick}

Die Interpolation der Linien erfolgt linear fuer die Start-
und Endpunkte. Wie beschrieben werden Rotationen durch
diese Weise skaliert, auch wenn die laenge des Linienpaares
jeweils gleich bleibt. Beier und Neely beschreiben,
dass eine weitere Moeglichkeit zur Interpolation folgendermassen
aussieht:
Die Mittelpunkte der beiden Linien und deren Orientierung
werden interpoliert. Denkbar waere hierbei die Nutzung
von Quaternions, welche die Orientierungen der beiden
Linien repraesentieren. Das Erzeugen aus Winkel
und Vektor wird durch \textbf{GLM} unterstuetzt:

\begin{lstlisting}[language=C++, caption=Quaternions in GLM, label=quaternion, xleftmargin=0.5cm]

glm::quat qSource = glm::angleAxis(theta, glm::vec3(0, 0, 1));
\end{lstlisting}
qSource repraesentiert nun eine Drehung um die Z-Achse
(unsere Linien werden im R2 platziert, wodurch die
3. Dimension hinzugezogen wird, um einen weiteren
Freiheitsgrad zu erlangen).

Den Winkel $\theta$ erhaelt man durch:

$\theta = \arcsin \left(\frac{y}{\|\textbf{v}\|}\right)$

wobei $y$ die 2. Komponente des Vektors $\textbf{v}$ ist, welcher die Linie in Ziel- bzw. Quellbild repraesentiert.
Die Interpolation zwischen zwei Quaternions erfolgt schliesslich durch:
\begin{lstlisting}[language=C++, caption=Spherical interpolation zwischen zwei Quaternions, label={lst:slerp}, xleftmargin=0.5cm]
	
	glm::quat qInterpolated = glm::slerp(qSource, qDest, 0.5);
	
\end{lstlisting}
In Listing \ref{lst:slerp} wird die Sphaerische Interpolation
zweier Quaternions bei $50\%$ berechnet.
glm::slerp sorgt dafuer, dass der kuerzeste Pfad zwischen
den beiden Orientierungen genommen wird. Um aus dem nun
interpolierten Quaternion wieder einen Vektor (und damit
eine Linie) zu erzeugen verdeutlicht man sich, dass
ein Quaternion eine Rotation repraesentiert. Wir
rotieren den Quellvektor (die Quelllinie)




