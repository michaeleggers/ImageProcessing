\chapter{Software}

Fuer den Kurs haben wir eine Beispielanwendung, PowerMorph, erstellt.
Fuer das Bauen wird ein CMake file bereitgestellt. Die Software basiert
auf OpenGL 3.0 und ist somit auch auf MacOS lauffaehig (zwar hat Apple
den Support von OpenGL eingestellt, allerdings werden Anwendungen, welche bis
einschl. maximal Version 4.1 von OpenGL verwenden, nach wie vor unterstuetzt).
Ausserdem nutzen wir folgende externe Bibliotheken:

\begin{itemize}
	\item \textbf{SDL2} als Abstraktion zum Betriebssystem fuer Fenster und OpenGL context Erstellung.
	\href{https://github.com/libsdl-org/SDL}{https://github.com/libsdl-org/SDL} 
	\item \textbf{STB image/image write}: Lesen/Schreiben von Bilddateien.
		\href{https://github.com/nothings/stb}{https://github.com/nothings/stb} 
	\item \textbf{GLM}: Mathematik Bibliothek, die gut mit OpenGL zusammenarbeitet. 
			\href{https://github.com/g-truc/glm}{https://github.com/g-truc/glm}
	\item \textbf{Dear ImGUI}: Immediate Mode GUI Bibliothek fuer den Editor.
	\item \textbf{tinyfiledialogs}: Betriebssystemunabhaengige Bibliothek fuer Window-Messages, Oeffnen/Speichern Dialoge.
	\item \textbf{GIF writer by Charlie Tangora}: Speichern der gerenderten Sequenzen als GIF.
				\href{https://github.com/charlietangora/gif-h}{https://github.com/charlietangora/gif-h} 
\end{itemize}

\section{Installation}

Wir nutzen CMake, um Projektdateien
fuer eine IDE bzw. Makefiles zu generieren. 

Nach der Installation ist sollte die Software ohne weiteres Aufrufbar sein. Ein fertiges Binary fuer Windows 64 bit ist auf 
www.nocheinfuegen.de herunterladbar.

